%%%%%%%%%%%%%%%%%%%%%%%%%%%%%%%%%%%%%%%%%%%%%%%%%%%%%%%%%%%%%%%%%%%%%%%%%%%%
%
%A  basic.tex            DESIGN documentation              Leonard Soicher
%
%
%
\def\DESIGN{\sf DESIGN}
\def\GRAPE{\sf GRAPE}
\def\nauty{\it nauty}
\def\Aut{{\rm Aut}\,}
\def\x{\times}
\Chapter{Determining basic properties of block designs}

%%%%%%%%%%%%%%%%%%%%%%%%%%%%%%%%%%%%%%%%%%%%%%%%%%%%%%%%%%%%%%%%%%%%%%%%
\Section{The functions for basic properties}

\>IsBlockDesign( <obj> )

This boolean function  returns `true' if and only if <obj>, which can be
an object of arbitrary type, is a block design.

\beginexample
gap> IsBlockDesign(5);
false
gap> IsBlockDesign( BlockDesign(2,[[1],[1,2],[1,2]]) );
true
\endexample



\>IsBinaryBlockDesign( <D> )

This boolean function  returns `true' if and only if the block design
<D> is *binary*, that is, if no block of <D> has a repeated element.

\beginexample
gap> IsBinaryBlockDesign( BlockDesign(2,[[1],[1,2],[1,2]]) );
true
gap> IsBinaryBlockDesign( BlockDesign(2,[[1],[1,2],[1,2,2]]) );
false
\endexample



\>IsSimpleBlockDesign( <D> )

This boolean function  returns `true' if and only if the block design
<D> is *simple*, that is, if no block of <D> is repeated.

\beginexample
gap> IsSimpleBlockDesign( BlockDesign(2,[[1],[1,2],[1,2]]) );  
false
gap> IsSimpleBlockDesign( BlockDesign(2,[[1],[1,2],[1,2,2]]) );
true
\endexample



\>IsConnectedBlockDesign( <D> )

This boolean function  returns `true' if and only if the block design
<D> is *connected*, that is, if its incidence graph is a connected
graph.

\beginexample
gap> IsConnectedBlockDesign( BlockDesign(2,[[1],[2]]) ); 
false
gap> IsConnectedBlockDesign( BlockDesign(2,[[1,2]]) );  
true
\endexample



\>ReplicationNumber( <D> )
 
If the block design <D> is equireplicate, then this function returns
its replication number; otherwise `fail' is returned.

A block design $D$ is *equireplicate* with *replication number* $r$ if,
for every point $x$ of $D$, $r$ is equal to the sum over the blocks of
the multiplicity of $x$ in a block. For a binary block design this is
the same as saying that each point $x$ is contained in exactly $r$ blocks.

\beginexample
gap> ReplicationNumber(BlockDesign(4,[[1],[1,2],[2,3,3],[4,4]]));
2
gap> ReplicationNumber(BlockDesign(4,[[1],[1,2],[2,3],[4,4]]));  
fail
\endexample



\>BlockSizes( <D> )

This function returns the set of sizes (actually list-lengths) of the 
blocks of the block design <D>.

\beginexample
gap> BlockSizes( BlockDesign(3,[[1],[1,2,2],[1,2,3],[2],[3]]) );  
[ 1, 3 ]
\endexample



\>BlockNumbers( <D> )

Let <D> be a block design. Then this function returns a list of
the same length as `BlockSizes(<D>)', such that the $i$-th element 
of this returned list is the number of blocks of <D> of size
`BlockSizes(<D>)[$i$]'.

\beginexample
gap> D:=BlockDesign(3,[[1],[1,2,2],[1,2,3],[2],[3]]); 
rec( isBlockDesign := true, v := 3, 
  blocks := [ [ 1 ], [ 1, 2, 2 ], [ 1, 2, 3 ], [ 2 ], [ 3 ] ] )
gap> BlockSizes(D);
[ 1, 3 ]
gap> BlockNumbers(D);
[ 3, 2 ]
\endexample



\>TSubsetLambdasVector( <D>, <t> )

Let <D> be a block design, <t> a non-negative integer, and 
`$v$=<D>.v'. Then this function returns an integer vector $L$ 
whose positions correspond to the <t>-subsets of $\{1,\ldots,v\}$.
The $i$-th element of $L$ is the sum over all blocks $B$ of <D> 
of the number of times the $i$-th <t>-subset (in lexicographic order) 
is contained in $B$. (For example, if $t=2$ and $B=[1,1,2,3,3,4]$, then
$B$ contains $[1,2]$ twice, $[1,3]$ four times, $[1,4]$ twice,
$[2,3]$ twice, $[2,4]$ once, and $[3,4]$ twice.) In particular, 
if <D> is binary then $L[i]$ is simply the number of blocks of <D> 
containing the $i$-th <t>-subset (in lexicographic order).
 
\beginexample
gap> D:=BlockDesign(3,[[1],[1,2,2],[1,2,3],[2],[3]]);;
gap> TSubsetLambdasVector(D,0);
[ 5 ]
gap> TSubsetLambdasVector(D,1);
[ 3, 4, 2 ]
gap> TSubsetLambdasVector(D,2);
[ 3, 1, 1 ]
gap> TSubsetLambdasVector(D,3);
[ 1 ]
\endexample



\>AllTDesignLambdas( <D> )

If the block design <D> is not a $t$-design for some $t\ge 0$ then this
function returns an empty list. Otherwise <D> is a binary block design
with constant block size $k$, say, and this function returns a list
$L$ of length $T+1$, where $T$ is the maximum $t\le k$ such that <D>
is a $t$-design, and, for $i=1,\ldots,T+1$, $L[i]$ is equal to the
(constant) number of blocks of <D> containing an $(i-1)$-subset of
$\{1,\ldots,`<D>.v'\}$.

\beginexample
gap> AllTDesignLambdas(PGPointFlatBlockDesign(3,2,1));                  
[ 35, 7, 1 ]
\endexample


%%%%%%%%%%%%%%%%%%%%%%%%%%%%%%%%%%%%%%%%%%%%%%%%%%%%%%%%%%%%%%%%%%%%%%%%
