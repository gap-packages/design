%%%%%%%%%%%%%%%%%%%%%%%%%%%%%%%%%%%%%%%%%%%%%%%%%%%%%%%%%%%%%%%%%%%%%%%%%%%%
%
%A  partition.tex        DESIGN documentation           Leonard Soicher
%
%
%
\def\GRAPE{\sf GRAPE}
\def\DESIGN{\sf DESIGN}
\def\nauty{\it nauty}
\def\Aut{{\rm Aut}\,} 

\Chapter{Partitioning block designs}

This chapter describes the function `PartitionsIntoBlockDesigns' which can
classify partitions of (the block multiset of) a given block design into
(the block multisets of) block designs having user-specified properties.
We also describe `MakeResolutionsComponent' which is useful for the
special case when the desired partitions are resolutions.

%%%%%%%%%%%%%%%%%%%%%%%%%%%%%%%%%%%%%%%%%%%%%%%%%%%%%%%%%%%%%%%%%%%%%%%%
\Section{Partitioning a block design into block designs}

\>PartitionsIntoBlockDesigns( <param> )

Let <D> equal `<param>.blockDesign'.  This function returns a list <PL>
of partitions of (the block multiset of) <D>.  Each element of <PL> is a
record with one component `partition', and, in most cases, a component
`autGroup'.  The `partition' component gives a list <P> of block designs,
all with the same point set as <D>, such that the list of (the block
multisets of) the designs in `<P>.partition' forms a partition of (the
block multiset of) <D>. The component `<P>.autGroup', if bound, gives
the automorphism group of the partition, which is the stabilizer of the
partition in the automorphism group of <D>.  The precise interpretation
of the output depends on <param>, described below.

The required components of <param> are `blockDesign', `v', `blockSizes',
and `tSubsetStructure'.

`<param>.blockDesign' is the block design to be partitioned.

`<param>.v' must be a positive integer, and specifies that for each block
design in each partition in <PL>, the points are 1,...,`<param>.v'.
It is required that `<param>.v' be equal to `<param>.blockDesign.v'.

`<param>.blockSizes' must be a set of positive integers, and specifies
that the block sizes of each block design in each partition in <PL>
will be contained in `<param>.blockSizes'.

`<param>.tSubsetStructure' must be a record, having
components `t', `partition', and `lambdas'. Let <t> be
equal to `<param>.tSubsetStructure.t', <partition> be
`<param>.tSubsetStructure.partition', and <lambdas> be
`<param>.tSubsetStructure.lambdas'.  Then <t> must be a non-negative
integer, <partition> must be a list of non-empty sets of <t>-subsets of
`[1..<param>.v]', forming an ordered partition of all the <t>-subsets of
`[1..<param>.v]', and <lambdas> must be a list of distinct non-negative
integers (not all zero) of the same length as <partition>. This specifies
that for each design in each partition in <PL>, each <t>-subset in
`<partition>[<i>]' will occur exactly `<lambdas>[<i>]' times, counted
over all blocks of the design.  For binary designs, this means that each
<t>-subset in `<partition>[<i>]' is contained in exactly `<lambdas>[<i>]'
blocks.  The `partition' component is optional if <lambdas> has length 1.
We require that <t> is less than or equal to each element of
`<param>.blockSizes', and that each block of `<param>.blockDesign'
contains at least <t> distinct elements.

The optional components of <param> are used to specify additional
constraints on the partitions in <PL>, or to change default parameter
values. These optional components are `r', `b', `blockNumbers',
`blockIntersectionNumbers', `blockMaxMultiplicities', `isoGroup',
`requiredAutSubgroup', and `isoLevel'. Note that the last three of these
optional components refer to the partitions and not to the block designs
in a partition.

`<param>.r' must be a positive integer, and specifies that in each design
in each partition in <PL>, each point must occur exactly `<param>.r'
times in the list of blocks.

`<param>.b' must be a positive integer, and specifies that each design
in each partition in <PL> has exactly `<param>.b' blocks.

`<param>.blockNumbers' must be a list of non-negative integers, the <i>-th
element of which specifies the number of blocks whose size is equal to
`<param>.blockSizes[<i>]' (in each design in each partition in <PL>). The
length of `<param>.blockNumbers' must equal that of `<param>.blockSizes',
and at least one entry of `<param>.blockNumbers' must be positive.

`<param>.blockIntersectionNumbers' must be a symmetric matrix of sets
of non-negative integers, the `[<i>][<j>]'-element of which specifies
the set of possible sizes for the intersection of a block of size
`<param>.blockSizes[<i>]' with one of size `<param>.blockSizes[<j>]'
(in each design in each partition in <PL>). In the case of multisets, we
take the multiplicity of an element in the intersection to be the minimum
of its multiplicities in the multisets being intersected, so, for example,
the intersection of `[1,1,1,2,2,3]' with `[1,1,2,2,2,4]' is `[1,1,2,2]',
having size 4.  The dimension of `<param>.blockIntersectionNumbers'
must equal the length of `<param>.blockSizes'.

`<param>.blockMaxMultiplicities' must be a list of non-negative
integers, the <i>-th element of which specifies an upper bound on the
multiplicity of a block whose size is equal to `<param>.blockSizes[<i>]'
(for each design in each partition in <PL>). The length of
`<param>.blockMaxMultiplicities' must equal that of `<param>.blockSizes'.

`<param>.isoGroup' must be a subgroup of the automorphism group of
`<param>.blockDesign'. We consider two elements of <PL> to be
*equivalent* if they are in the same orbit of `<param>.isoGroup'
(in its action on multisets of block multisets).  The default for
`<param>.isoGroup' is the automorphism group of `<param>.blockDesign'.

`<param>.requiredAutSubgroup' must be a subgroup of `<param>.isoGroup',
and specifies that each partition in <PL> must be invariant under
`<param>.requiredAutSubgroup' (in its action on multisets of block
multisets). The default for `<param>.requiredAutSubgroup' is the trivial
permutation group.

`<param>.isoLevel' must be 0, 1, or 2 (the default is 2).  The value 0
specifies that <PL> will contain at most one partition, and will contain
one partition with the required properties if and only if such a partition
exists; the value 1 specifies that <PL> will contain (perhaps properly)
a list of `<param>.isoGroup' orbit-representatives of the required
partitions; the value 2 specifies that <PL> will consist precisely of
`<param>.isoGroup'-orbit representatives of the required partitions.

For an example, we first classify up to isomorphism the 2-(15,3,1)
designs invariant under a semi-regular group of automorphisms of order 5,
and then use `PartitionsIntoBlockDesigns' to classify all the resolutions
of these designs, up to the actions of the respective automorphism groups
of the designs.

\beginexample
gap> DL:=BlockDesigns(rec(
>    v:=15,blockSizes:=[3],
>    tSubsetStructure:=rec(t:=2,lambdas:=[1]),
>    requiredAutSubgroup:=
>       Group((1,2,3,4,5)(6,7,8,9,10)(11,12,13,14,15))));;
gap> List(DL,D->Size(AutGroupBlockDesign(D)));
[ 20160, 5, 60 ]
gap> PL:=PartitionsIntoBlockDesigns(rec(
>       blockDesign:=DL[1],
>       v:=15,blockSizes:=[3],
>       tSubsetStructure:=rec(t:=1,lambdas:=[1])));
[ rec( 
      partition := [ rec( isBlockDesign := true, v := 15, blocks := [ [ 1, 2, 
                      6 ], [ 3, 4, 8 ], [ 5, 7, 14 ], [ 9, 12, 15 ], 
                  [ 10, 11, 13 ] ] ), 
          rec( isBlockDesign := true, v := 15, blocks := 
                [ [ 1, 3, 11 ], [ 2, 4, 12 ], [ 5, 6, 8 ], [ 7, 13, 15 ], 
                  [ 9, 10, 14 ] ] ), 
          rec( isBlockDesign := true, v := 15, blocks := 
                [ [ 1, 4, 14 ], [ 2, 5, 15 ], [ 3, 10, 12 ], [ 6, 7, 11 ], 
                  [ 8, 9, 13 ] ] ), 
          rec( isBlockDesign := true, v := 15, blocks := 
                [ [ 1, 5, 10 ], [ 2, 9, 11 ], [ 3, 14, 15 ], [ 4, 6, 13 ], 
                  [ 7, 8, 12 ] ] ), 
          rec( isBlockDesign := true, v := 15, blocks := 
                [ [ 1, 7, 9 ], [ 2, 8, 10 ], [ 3, 5, 13 ], [ 4, 11, 15 ], 
                  [ 6, 12, 14 ] ] ), 
          rec( isBlockDesign := true, v := 15, blocks := 
                [ [ 1, 8, 15 ], [ 2, 13, 14 ], [ 3, 6, 9 ], [ 4, 7, 10 ], 
                  [ 5, 11, 12 ] ] ), 
          rec( isBlockDesign := true, v := 15, blocks := 
                [ [ 1, 12, 13 ], [ 2, 3, 7 ], [ 4, 5, 9 ], [ 6, 10, 15 ], 
                  [ 8, 11, 14 ] ] ) ], 
      autGroup := Group([ (1,10)(2,11)(3,8)(6,13)(7,14)(12,15), 
          (1,13)(2,11)(3,14)(4,5)(6,10)(7,8), 
          (1,13,7)(2,11,5)(6,10,14)(9,12,15), 
          (2,11,5,15,4,9,12)(3,10,8,14,7,13,6) ]) ), 
  rec( partition := [ rec( isBlockDesign := true, v := 15, 
              blocks := [ [ 1, 2, 6 ], [ 3, 4, 8 ], [ 5, 7, 14 ], 
                  [ 9, 12, 15 ], [ 10, 11, 13 ] ] ), 
          rec( isBlockDesign := true, v := 15, 
              blocks := [ [ 1, 3, 11 ], [ 2, 4, 12 ], [ 5, 6, 8 ], 
                  [ 7, 13, 15 ], [ 9, 10, 14 ] ] ), 
          rec( isBlockDesign := true, v := 15, 
              blocks := [ [ 1, 4, 14 ], [ 2, 5, 15 ], [ 3, 10, 12 ], 
                  [ 6, 7, 11 ], [ 8, 9, 13 ] ] ), 
          rec( isBlockDesign := true, v := 15, 
              blocks := [ [ 1, 5, 10 ], [ 2, 13, 14 ], [ 3, 6, 9 ], 
                  [ 4, 11, 15 ], [ 7, 8, 12 ] ] ), 
          rec( isBlockDesign := true, v := 15, 
              blocks := [ [ 1, 7, 9 ], [ 2, 8, 10 ], [ 3, 14, 15 ], 
                  [ 4, 6, 13 ], [ 5, 11, 12 ] ] ), 
          rec( isBlockDesign := true, v := 15, 
              blocks := [ [ 1, 8, 15 ], [ 2, 9, 11 ], [ 3, 5, 13 ], 
                  [ 4, 7, 10 ], [ 6, 12, 14 ] ] ), 
          rec( isBlockDesign := true, v := 15, 
              blocks := [ [ 1, 12, 13 ], [ 2, 3, 7 ], [ 4, 5, 9 ], 
                  [ 6, 10, 15 ], [ 8, 11, 14 ] ] ) ], 
      autGroup := Group([ (1,15)(2,9)(3,4)(5,7)(6,12)(10,13), 
          (1,12)(2,9)(3,5)(4,7)(6,15)(8,14), 
          (1,14)(2,5)(3,8)(6,7)(9,12)(10,13), 
          (1,8,10)(2,5,15)(3,14,13)(4,9,12) ]) ) ]
gap> List(PL,resolution->Size(resolution.autGroup));
[ 168, 168 ]
gap> PL:=PartitionsIntoBlockDesigns(rec(
>       blockDesign:=DL[2],
>       v:=15,blockSizes:=[3],
>       tSubsetStructure:=rec(t:=1,lambdas:=[1])));
[  ]
gap> PL:=PartitionsIntoBlockDesigns(rec(
>       blockDesign:=DL[3],
>       v:=15,blockSizes:=[3],
>       tSubsetStructure:=rec(t:=1,lambdas:=[1])));
[  ]
\endexample

%%%%%%%%%%%%%%%%%%%%%%%%%%%%%%%%%%%%%%%%%%%%%%%%%%%%%%%%%%%%%%%%%%%%%%%%
\Section{Computing resolutions}

\>MakeResolutionsComponent( <D> )
\>MakeResolutionsComponent( <D>, <isolevel> )

This function computes resolutions of the block design <D>, and stores
the result in `<D>.resolutions'. If `<D>.resolutions' already exists
then it is ignored and overwritten. This function returns no value.
 
A *resolution* of a block design $D$ is a partition of the blocks into
subsets, each of which forms a partition of the point set.  We say that
two resolutions $R$ and $S$ of $D$ are *isomorphic* if there is an element
$g$ in the automorphism group of $D$, such that the $g$-image of $R$
is $S$. (Isomorphism defines an equivalence relation on the set of
resolutions of $D$.)

The parameter <isolevel> (default 2) determines how many resolutions are
computed: <isolevel>=2 means to classify up to isomorphism, <isolevel>=1
means to determine at least one representative from each isomorphism
class, and <isolevel>=0 means to determine whether or not <D> has
a resolution.

When this function is finished, `<D>.resolutions' will have the following
three components:

`list': a list of distinct partitions into block designs forming resolutions
of <D>;

`pairwiseNonisomorphic': `true', `false' or `"unknown"', depending on the
resolutions in `list' and what is known. If <isolevel>=0 or <isolevel>=2
then this component will be `true';

`allClassesRepresented': `true', `false' or `"unknown"', depending on the
resolutions in `list' and what is known. If <isolevel>=1 or <isolevel>=2
then this component will be `true'.

Note that `<D>.resolutions' may be changed to contain more information
as a side-effect of other functions in the {\DESIGN} package.

\beginexample
gap> L:=BlockDesigns(rec(v:=9,blockSizes:=[3],
>          tSubsetStructure:=rec(t:=2,lambdas:=[1])));;
gap> D:=L[1];;
gap> MakeResolutionsComponent(D);
gap> D;
rec( isBlockDesign := true, v := 9, 
  blocks := [ [ 1, 2, 3 ], [ 1, 4, 5 ], [ 1, 6, 7 ], [ 1, 8, 9 ], 
      [ 2, 4, 6 ], [ 2, 5, 8 ], [ 2, 7, 9 ], [ 3, 4, 9 ], [ 3, 5, 7 ], 
      [ 3, 6, 8 ], [ 4, 7, 8 ], [ 5, 6, 9 ] ], 
  tSubsetStructure := rec( t := 2, lambdas := [ 1 ] ), isBinary := true, 
  isSimple := true, blockSizes := [ 3 ], blockNumbers := [ 12 ], r := 4, 
  autGroup := Group([ (1,2)(5,6)(7,8), (1,3,2)(4,8,7)(5,6,9), (1,2)(4,7)(5,9),
      (1,2)(4,9)(5,7)(6,8), (1,4,8,6,9,2)(3,5,7) ]), 
  resolutions := rec( list := [ rec( partition := 
                [ rec( isBlockDesign := true, v := 9, 
                      blocks := [ [ 1, 2, 3 ], [ 4, 7, 8 ], [ 5, 6, 9 ] ] ), 
                  rec( isBlockDesign := true, v := 9, 
                      blocks := [ [ 1, 4, 5 ], [ 2, 7, 9 ], [ 3, 6, 8 ] ] ), 
                  rec( isBlockDesign := true, v := 9, 
                      blocks := [ [ 1, 6, 7 ], [ 2, 5, 8 ], [ 3, 4, 9 ] ] ), 
                  rec( isBlockDesign := true, v := 9, 
                      blocks := [ [ 1, 8, 9 ], [ 2, 4, 6 ], [ 3, 5, 7 ] ] ) ],
              autGroup := Group(
                [ (2,3)(4,5)(6,7)(8,9), (1,3,2)(4,8,7)(5,6,9), 
                  (1,8,9)(2,4,6)(3,7,5), (1,2)(5,6)(7,8), (1,2)(4,7)(5,9), 
                  (1,2,9,6,8,4)(3,7,5) ]) ) ], pairwiseNonisomorphic := true, 
      allClassesRepresented := true ) )
\endexample

%%%%%%%%%%%%%%%%%%%%%%%%%%%%%%%%%%%%%%%%%%%%%%%%%%%%%%%%%%%%%%%%%%%%%%%%
